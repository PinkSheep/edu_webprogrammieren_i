\documentclass[10pt,ngerman]{examdesign}
\usepackage[utf8]{inputenc}
\usepackage{listings}
\usepackage{graphicx}

\Fullpages
\NumberOfVersions{2}
\SectionPrefix{Abschnitt \arabic{sectionindex}. \space}
\lstset{language=Java,numbers=left}
\IncludeFromFile{listings.tex}

\begin{document}

\begin{examtop}
    \parbox{5cm}{
        Pr\"ufung \\
        Webprogrammieren I\\
        Lafon \\
        }
    \hfill
    \parbox{10cm}{
    Name: \hrulefill\\[6pt]
    Klasse: \hrulefill\\[6pt]
    Datum: \hrulefill}
    \bigskip

\begin{itemize}
 \item Markieren Sie richtige Antworten mit einem Kreuz.
 \item Falls Sie sich umentscheiden sollten und eine bereits als richtig
 markierte Antwort trotzdem als nicht richtig bezeichnen m\"ochten, machen Sie
 einen Kreis um das Kreuz.
 \item Pro Frage k\"onnen keine, eine oder mehrere Antworten richtig sein.
 \item Antworten welche als richtig markiert wurden, jedoch falsch sind, geben
 einen Punkt Abzug.
 \item Wenn von einem Programm oder einer Software die Rede ist, dann sind
 Programme von gr\"osserer Komplexit\"at gemeint (z.B. analog Ihrer Fallstudie,
 keine Hello-World Programme).
 \item Falls Ihnen eine Frage nicht klar sein sollte, treffen Sie eine Annahme
 und halten Sie diese schriftlich fest.
 \item Sämtliche Arbeiten sind absolut selbständig auszuführen. Wer sich unredlich verhält,
wird unmittelbar nach Feststellung des Tatbestandes von der Prüfung ausgeschlossen.
In diesem Fall  gilt die ganze Prüfung als nicht bestanden.  Bei  besonders
schwerwiegender Unredlichkeit wird der fehlbare Studierende aus der Schule
ausgeschlossen.
\end{itemize}

\end{examtop}

\begin{multiplechoice}

  \begin{question}
    Welche Eigenschaften bringt man mit DNS in Verbindung?
    \choice[!]{"Hierarchisch"}
    \choice[!]{"Redundant"}
    \choice[!]{"Dezentral"}
    \choice[]{"Flach"}
    \smallskip
  \end{question}


  \begin{question}
    Wie heisst die Transportschicht im Browser?
    \choice[]{"SOAP"}
    \choice[!]{"HTTP"}
    \choice[]{"WebDAV"}
    \choice[]{"REST"}
    \smallskip
  \end{question}


  \begin{question}
    Was sind die gängigen Ports für HTTP(S)?
    \choice[!]{"80"}
    \choice[]{"21"}
    \choice[]{"23"}
    \smallskip
  \end{question}


  \begin{question}
    Welche Tags sind semantisch?
    \choice[!]{"q"}
    \choice[]{"div"}
    \choice[!]{"a"}
    \smallskip
  \end{question}


  \begin{question}
    Welche Tags sind inline(i), welche block(b) Elemente?
    \answer{b}{"div"}
    \answer{i}{"span"}
    \answer{i}{"q"}
    \answer{b}{"p"}
    \answer{i}{"a"}
    \smallskip
  \end{question}


  \begin{question}
    Nennen Sie stichwortartig Vorteile der besseren Code Beispiels!
    \InsertChunk{semantic_html}
    \InsertChunk{non_semantic_html}
    \bigskip
  \end{question}

  \begin{question}
    Notieren Sie die Syntax von URL!
    \bigskip
  \end{question}

  \begin{question}
    Wof\"ur steht URI?
    \choice[]{"Unique Resource Indicator"}
    \choice[]{"Uniform Resource Locator"}
    \choice[!]{"Uniform Resource Identifier"}
    \choice[!]{"Das ist in Kanton in der Schweiz"}
    \smallskip
  \end{question}

  \begin{question}
    Ihr Kunde m\"ochte eine Webseite haben mit einem animierten
    Baustellenschild dessen Tooltip lautet: "Hier ensteht eine neue
    Internetpr\"asenz!". Beim Click auf das Baustellenschild soll sich ein
    neues Browserfenster mit der alten Webseite des Kunden(c2.com) \"offnen.
    Schreiben Sie den dazu notwendigen Inhalt des HTML \<body\>!
    \bigskip
  \end{question}

  \begin{question}
    Ihr Kunde m\"ochte nun den Hintergrund blau und das Baustellenschild
    zentriert! Schreiben Sie dazu den notwendigen HTML und CSS Code!
    \bigskip
  \end{question}

  \begin{question}
    Bringen Sie die folgenden Technologien in die Reihenfolge der jeweiligen
    Entstehung!
    \answer{1}{"SGML"}
    \answer{4}{"HTML5"}
    \answer{3}{"XHTML1"}
    \answer{2}{"HTML4"}
    \smallskip
  \end{question}

  \begin{question}
    Wozu dient 'charset' im 'content' Attribut des 'meta' Tags?
    \bigskip
  \end{question}

  \begin{question}
    \InsertChunk{greyish}
    Was haben die folgenden Farben gemein?
    \bigskip
  \end{question}

  \begin{question}
    \InsertChunk{redish}
    Was haben die folgenden Farben gemein?
    \bigskip
  \end{question}

  \begin{question}
    Bringen Sie die Begriffe 'padding', 'content', 'margin', 'border'  in Bezug
    auf das CSS Box Modell in die richtige Reihenfolge
    \answer{1}{"content"}
    \answer{2}{"padding"}
    \answer{3}{"border"}
    \answer{4}{"margin"}
    \smallskip
  \end{question}


  %TODO! (padding und so)
  \begin{question}
    Skizzieren Sie wie der Browser folgenden Quellcode rendert!
    \InsertChunk{redish}
    \InsertChunk{redish}
    \bigskip
  \end{question}

  \begin{question}
    Wann setzt man CSS Pseudo Klassen ein?
    \bigskip
    %\choice[!]{"Verschiedene Zust\"ande eines Elementes"}
    %\choice[!]{"Algorithmische Selektoren(ehemals Javascript)"}
  \end{question}

  % TODO: Mache HTML/CSS mit überall inline Style => Implizites Abfragen von
  % "!important"
  \begin{question}
    \"Andern Sie den Quellcode so, dass 'foo' rot dargestellt wird.
    \bigskip
  \end{question}

  \begin{question}
    Worin unterscheidet sich das Verhalten von 'margin' bei inline und block
    Elementen?
    \bigskip
    % inline: margins werden addiert, block: margins werden kollabiert
  \end{question}

  \begin{question}
    Welche Aussagen treffen auf die Positionierung 'static'(s), 'absolute'(a),
    'relative'(r), 'fixed'(f) zu?
    \answer{s} Elemente bewegen sich immer relativ zum normalen Fluss der Seite
    \answer{f} Elemente bewegen sich nicht, selbst wenn das Fenster scrollt
    \answer{r} Elemente ist verh\"altnissm\"assig zu seiner normalen Position
    positioniert
    \answer{a} Elemente sind relativ zum ersten \"Ubergeordneten Element
    positioniert, welches nicht 'static' ist
    \smallskip
  \end{question}


  % TODO: Beschreibe Fibonacci Sequenz aus alter Vorlesung
  \begin{question}
    Summieren Sie alle geraden Zahlen der Fibonacci Sequenz von 1 bis 4
    Millionen.\\
    \bigskip
  \end{question}









  \begin{question}
    Schauen Sie folgenden erb-Codeblock an. Was ist die Ausgabe? (Nehmen wir an, es ist 12 Uhr mittags)
    \InsertChunk{erb}
    \choice[]{"It is now Wed Apr 13 12:00:00 +0200 2011"}
    \choice[]{"It is now 12:00:00"}
    \choice[!]{"It is now "}
    \choice[]{Ein Fehler tritt auf}
    \smallskip
  \end{question}

\end{multiplechoice}


        Ordnen Sie die folgenden Begriffen den Schichten M(odel), V(iew) oder
        C(ontrol) zu.
        \begin{question}
                        \answer{M} Datenbankzugriff \"uber SQL
        \end {question}
        \begin{question}
                        \answer{M} Datenzugriff \"uber ein XML Framework
        \end {question}
        \begin{question}
                        \answer{V}	Ausgabe der Daten im HTML Format
        \end {question}
        \begin{question}
                        \answer{V} Ausgabe der Daten in XML Format
        \end {question}
\end{truefalse}

\end{document}
